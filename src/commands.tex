\newcommand{\TODO}[1]{\textcolor{red}{TODO: #1}}

\addto\extrasenglish{%
  \renewcommand{\chapterautorefname}{Chapter}%
  \renewcommand{\sectionautorefname}{Sec.}%
  \renewcommand{\subsectionautorefname}{Sec.}%
  \renewcommand{\subsubsectionautorefname}{Sec.}%
  \renewcommand{\paragraphautorefname}{Par.}%
  \renewcommand{\tableautorefname}{Tab.}%
  \renewcommand{\equationautorefname}{Eq.}%
  \renewcommand{\figureautorefname}{Fig.}%
  \renewcommand{\appendixautorefname}{Appendix}%
  \renewcommand{\theoremautorefname}{Property}
}
\newcommand{\algorithmautorefname}{Alg.}

\newcommand{\atUL}[1]{\textcolor{blue}{@UL: #1}}
\newcommand{\atOTH}[1]{\textcolor{magenta}{@OTH: #1}}

\newcommand{\Ket}[1]{$\ket{#1}$}
\newcommand{\Bra}[1]{$\bra{#1}$}
\newcommand{\BraKet}[2]{$\braket{#1}{#2}$}

\makeatletter
\def\calcLength(#1,#2)#3{%
  \pgfpointdiff{\pgfpointanchor{#1}{center}}%
  {\pgfpointanchor{#2}{center}}%
  \pgf@xa=\pgf@x%
  \pgf@ya=\pgf@y%
  \FPeval\@temp@a{\pgfmath@tonumber{\pgf@xa}}%
  \FPeval\@temp@b{\pgfmath@tonumber{\pgf@ya}}%
  \FPeval\@temp@sum{(\@temp@a*\@temp@a+\@temp@b*\@temp@b)}%
  \FProot{\FPMathLen}{\@temp@sum}{2}%
  \FPround\FPMathLen\FPMathLen5\relax
  \global\expandafter\edef\csname #3\endcsname{\FPMathLen}
}
\makeatother

\lstdefinestyle{query}{
  language=SQL,
  stepnumber=1,
  numbersep=10pt,
  tabsize=4,
  showspaces=false,
  showstringspaces=false,
  basicstyle=\linespread{1}\fontfamily{lmtt}\selectfont\small,
  keywordstyle=\color{blue},
  stringstyle=\color{purple},
  upquote=true,
  breaklines=true,
  commentstyle=\color{CadetBlue}
}

\definecolor{mygray}{rgb}{0.643,0.643,0.643}
\newtcolorbox{querybox}[2][]{%
  sidebyside align=top,
  enhanced,
  boxsep=2pt,
  arc=0pt,
  top=-3pt, bottom=-3pt,
  left=2pt, right=0pt,
  colback=white,
  colframe=mygray,
  boxrule=0.5pt,
  leftrule=12pt,
  overlay unbroken and first ={%
      \node[rotate=90,
        minimum width=0.5cm,
        anchor=south,
        yshift=-11pt,
        white]
      at (frame.west) {#2};
    }
}

\newtcolorbox{matrixbox}[2][]{%
  sidebyside align=top,
  enhanced,
  boxsep=0pt,
  arc=0pt,
  left=-1em,
  top=-0.8em,
  boxrule=0pt,
  colframe=bggray,
  colback=bggray,
  leftrule=12pt,
  overlay unbroken and first ={%
      \node[rotate=90,
        minimum width=0.5cm,
        anchor=south west,
        font=\itshape,
        yshift=0pt, %-14pt,
        xshift=0.5em,
        black]
      at (frame.south west) {#2};
    }
}
\newcommand{\CGate}[1]{\ensuremath{\text{C\raisebox{0.08em}{--}}\!#1}\xspace}
\newcommand{\rx}{R_X}
\newcommand{\ry}{R_Y}
\newcommand{\rz}{R_Z}
\newcommand{\rzz}{R_{ZZ}}
\newcommand{\rzzT}{R_{Z^2}}
\newcommand{\rzn}{R_{Z^n}}

\DeclareMathOperator*{\argmax}{arg\,max}
\DeclareMathOperator*{\argmin}{arg\,min}

\newtheoremstyle{property}% 〈name〉
{3pt}% 〈Space above〉1
{3pt}% 〈Space below 〉1
{}% 〈Body font〉
{}% 〈Indent amount〉2
{\bfseries}% 〈Theorem head font〉
{:}% 〈Punctuation after theorem head 〉
{.5em}% 〈Space after theorem head 〉3
{}% 〈Theorem head spec (can be left empty, meaning ‘normal’ )〉

\theoremstyle{property}
\newtheorem{theorem}{Property} 
\newtheorem{corollary}{Corollary}[theorem] 
\newenvironment{subproof}[1][\proofname]{%
  \renewcommand{\qedsymbol}{$\blacksquare$}%
  \begin{proof}[#1]%
    }{%
  \end{proof}%
}

\newcommand{\ie}{\emph{i.e.}\xspace}
\newcommand{\eg}{\emph{e.g.}\xspace}
\newcommand{\etal}{\emph{et al.}\xspace}
\newcommand{\rj}{\emph{ReJoin}\xspace}

\newcommand{\suppweb}{\href{https://github.com/lfd/Fast_Quadratisation}{supplementary website}\xspace}
\newcommand{\repropkg}{\href{https://doi.org/10.5281/zenodo.14245587}{reproduction package}\xspace}

\newlength{\WIDTH}\newlength{\HEIGHT}
\renewcommand{\Join}{\bowtie}

\ifbool{anonymous}{}{\renewcommand{\censor}[1]{#1}}
\ifbool{anonymous}{}{\renewcommand{\blackout}[1]{#1}}
\ifbool{anonymous}{\newcommand{\genemail}[2]{\href{mailto:xxx.xxx@xx.xx}{\blackout{#2}}}}{\newcommand{\genemail}[2]{\href{#1}{#2}}}